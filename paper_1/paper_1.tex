\documentclass[12pt]{article}
\usepackage{amsmath, amsthm, amssymb, amsfonts}
\usepackage{hyperref}
\usepackage{graphicx}
\usepackage{float}
\usepackage{xcolor}

\DeclareMathOperator*{\argmin}{argmin}

%opening
\title{Flat Norm on Graphs}
\author{Sandy Auttelet, Jared Brannan, Blake Cecil, Curtis Michels,\\
 Katrina Sabochick, Kevin Vixie }

\begin{document}

\maketitle

\begin{abstract}
In this paper we implement and test a method for computing the multiscale flat norm signature for characteristic functions over irregular grids in $\mathbb{R}^2$ and $\mathbb{R}^3$.
\end{abstract}

\tableofcontents

\section{Multiscale Flat Norm}

In 2005, Chan and Esedoglu introduced an edge preserving total variation regularization functional:

\begin{equation} \label{ce}
F_{CE}(u,f) = \int_\Omega |\nabla u| dx + \lambda \int_{\Omega} |u-f|dx
\end{equation}

Where $\Omega$ is a domain on which we have greyscale data $f:\Omega \to \mathbb{R}$ we would like to denoise. Solving the associated minimization problem above we obtain:

\begin{align*}
u^* = \argmin_u F_{CE}(u,f)
\end{align*}

Where $u^*: \Omega \rightarrow \Omega$ is the denoised greyscale approximation to the data $f$. The strength of the denoising may be adjusted by the parameter $\lambda \in [0,\infty)$, a large value of $\lambda$ corresponds to enacting a stricter penalty for candidates $u$ that deviate too far from the original data and thus enforce less denoising. We henceforth refer to (\ref{ce}) as the $L^1$TV functional.

It was recognized in \cite{Morgan_2007} by Simon Morgan and Kevin Vixie that the $L^1$TV functional was both a special case of and an extension of the flat norm in geometric measure theory (GMT). The work done in \cite{shapes, hu_median, ibrahim_simplicial} explore the implications of this. These tools have subsequently found applications in chemistry \cite{flat_norm_chemistry_fingers}.


\section{Discrete Implementations}

Let $\chi_E$ is the characteristic function of $E$, with $\chi_E(x) = 1$ if $x \in E$ and $0$ otherwise. Then $u$ may be taken to be of this form, and the $L^1$TV functional in (\ref{ce}) reduces to:

\begin{align*}
F_{CE}(\Sigma,\Omega) &= \text{Per}(\Sigma) + \lambda|\Sigma \Delta \Omega|
\end{align*}

where $\Sigma$ is the support of $u = \chi_\Sigma$, Per($\Sigma$) is the perimeter of $\Sigma$, and $\Sigma \Delta \Omega$ is the symmetric difference between $\Sigma$ and the support $\Omega$ of the data $f = \chi_\Omega$ (see \cite{ce} for the details). 

The flat norm with scale $\lambda$ of an oriented $1$-dimensional set $T$ is given by:

\begin{equation} \label{fn}
\mathbb{F}_\lambda(T) = \min_S \{V_1(T-\partial S) + \lambda V_2(S)\}
\end{equation}

Where $V_1$ is 1-dimensional volume (length), $V_2$ is 2-dimensional volume (area) and $S$ varies over $2$-dimensional regions. We refer to the pair of the 1D and 2D sets $\{T,S^*\}$ as the flat norm decomposition, where $S^*$ is the minimizer of $(\ref{fn})$. In \cite{Morgan_2007} Morgan and Vixie established for $\Omega \subseteq \mathbb{R}^n$,

\begin{equation}
\mathbb{F}(\partial^* \Omega) = \min_{\Sigma} F_{CE}(\Sigma,\Omega)
\end{equation}

Where $\partial^*$ denotes the reduced boundary of $\Omega$ (up to a set of $H^{n-1}$ measure zero $\partial^* \Omega$ is equal to the measure theoretic boundary of $\Omega$.

Practically, Vixie and Morgan \cite{shapes} tells us that we can use any method that solves the Chan Esedoglu problem to compute the flatnorm for co-dimension 1 boundaries. Subsequently they used the graph cut method introduced in \cite{kolmogorov}. Additional methods based on simplicial complexes have been developed since that  extend the multiscale flat norm to inputs that are not necessarily boundaries \cite{ibrahim_simplicial}.

Applied to images, this is realized by representing each pixel as a node on a rectangular grid which forms the working space. A characteristic function $\chi_\Omega$ is defined on the nodes which represents a black and white thresholded image. Graph edges are added between each node using the 16 distinct directions in image space. Each of these edges are weighted by minimizing gradient computation error on known linear functions. After, a virtual sink ($t$) and a virtual source node ($s$) are added. The source node is connected to every node in $\Omega$ and the sink to every node on the grid not in $\Omega$. 

With the correct setting of weights a cut of this graph has a capacity equal to $F_{CE}(\Sigma,\Omega)$. For $\Sigma$ treated as the set of nodes in the graph that are either (1) connected to the source by an edge not cut or (2) are connected ot the sink by an edge that is cut, any cut of the graph incurs a penalty equal to $F_{CE}(\Sigma,\Omega)$ Hence finding a cut with minimal capacity is the same as computing a discrete approximation to $F_{CE}(\Sigma,\Omega)$ and hence an approximation to the flat norm as well. 

\begin{figure}[H]
	\centering
	\includegraphics[scale=1]{graph-cut-for-a-1-dimensional-image.png}
	\caption{A discrete approximation to $F_{CE}(\Sigma,\Omega)$ using graph cuts from \cite{Morgan_2007}.}
\end{figure}

The vector of weights $w^*$ calculated in \cite{shapes} were chosen more specifically to approximate the linear function $g_\theta:\mathbb{R}^2 \to \mathbb{R}$ whose gradient is $\nabla g_\theta = (\cos \theta, \sin \theta)^T$ for all $\theta$:

\begin{equation} \label{oldmin}
w^* = \argmin_w \int_0^{2\pi} (h(w,\theta)-1)^2 d\theta
\end{equation}

With

\begin{align*}
h(w,\omega) &= \sum_{j=1}^4 w_1 |\nabla g_\theta \cdot v_j| + \sum_{j=5}^8 w_2 |\nabla g_\theta \cdot v_j| +  \sum_{j=9}^{16} w_3 |\nabla g_\theta \cdot v_j|
\end{align*}

Where $v_j, j= 1,...,16$ are the vectors from a fixed point in the grid to its 16 nearest neighbors, where three types of neighbor groupings are identified as below.

\begin{figure}[H]
\centering
\includegraphics[scale=0.25]{Figure_2_Vixie_Paper.png}
\caption{16 vector neighborhood from \cite{shapes}.}
\end{figure}

Equation (\ref{oldmin}) was solved analytically to obtain weights $(w_1,w_2,w_3) \approx (0.1221,0.0476,0.0454)$. 


\section{Flat Norm on Arbitrary 2D and 3D Graphs}

In the present work, we extend the multiscale flat norm computation to arbitrary graphs embedded in $\mathbb{R}^n$ using the framework above, and provide code to calculate explicitly in $\mathbb{R}^2$ and $\mathbb{R}^3$.

Let $V = \{v_1,...,v_N\}$ be a set of vertices in $\mathbb{R}^n$ with a set of edges $E$ on $V$. Fix a particular vertex $v \in V$ with degree $D$ and associated edges  $\{u_i\}_{i=1}^D$. We provide a scheme for calculating the edge weights $\{w_i\}_{i=1}^D$ that recover the weights obtained in (\ref{oldmin}) in the case of a regular unit grid and extend to arbitrary irregular grids and connections. Our weights will be chosen to minimize the distance (in the $L^2$ sense) between the weighted sum of the edges and the length of the graident vector $\nu$ on $\partial B(0,1)$, the unit sphere at the origin:

\begin{align}
\mathcal{F}(w_1,w_2,...,w_D) &:= \int_{\partial B(0,1)} \left|\sum_{i=1}^D w_i |\langle \nu, u_i \rangle| - 1 \right|^2d\nu\\
&= \sum_{i=1}^D w_i^2C_1^i + 2 \sum_{i=1}^D \sum_{j=1}^{i-1} w_i w_j C_3^{ij} - 2 \sum_{i=1}^D w_i C_2^i + \alpha(n)
\end{align}

Where $\alpha(n) = \int_{\partial B(0,1)}d\nu$ is the measure of $\mathbb{S}^{n-1}$ in $\mathbb{R}^n$. 

\begin{align*}
C_1^i &:= \int_{\partial B(0,1)} |\langle \nu,u_i \rangle|^2 d\nu\\
C_2^i &:= \int_{\partial B(0,1)} |\langle \nu,u_i \rangle| d\nu\\
C_3^{ij} &:= \int_{\partial B(0,1)} |\langle \nu,u_i \rangle \langle \nu, u_j \rangle| d\nu
\end{align*}

\textcolor{red}{The function above is convex, and thus we minimize it by calculating the gradient}:

\begin{align*}
	\frac{\partial \mathcal{F}}{\partial \omega_k}  &= 2 \omega_k C_1^k + 2 \sum_{m=1}^{D}\omega_{m}C_{3}^{km} - 2\omega_kC_3^{kk} - 2 C_2^k\\
\end{align*}

Which forms a linear system of $D$ equations in $D$ unknowns. We then solve the following system for a stationary point:

\begin{align*}
2 \omega_1 C_1^1 + 2 \sum_{m=2}^{D}\omega_{m}C_{3}^{1m}  - 2 C_2^1 &= 0\\
\vdots \hspace{20mm} & \\
2 \omega_D C_1^D + 2 \sum_{m=1}^{D-1}\omega_{m}C_{3}^{Dm} - 2 C_2^D &= 0\\
\end{align*}

In practice this is done using least squares. If needed, the Hessian may also be quickly calculated:

\begin{align*}
\frac{\partial^2 \mathcal{F}}{\partial \omega_k \partial \omega_\ell} &= \begin{cases}
2C_1^k & \text{ if } \ell = k\\
2C_3^{k\ell} & \text{ otherwise}
\end{cases}
\end{align*}

Thus the problem reduces to calculation of $C_1^i$, $C_2^i$ and $C_3^{ij}$. We provide the following table of values:

\begin{figure}[H]
\centering
\begin{tabular}{|c|c|c|c|}
\hline
 &  2D  & 3D  \\
\hline
 $C_1^i$ & $\pi \| u_i \|^2$  & $\frac{4\pi}{3}\|u_i\|^2$   \\
\hline
 $C_2^i$ & $4\| u_i \|$  & $2\pi \|u_i\|$    \\
\hline
 $C_3^{ij}$ & Numerical  & Numerical    \\
\hline
\end{tabular}
\end{figure}

Solving this system produces a weight $w_i$ for each edge $u_i$ for the vertex $v$, in order to approximate the total variation integral in (5) we multiply each weight by the area of its associated Voronoi cell. With the weighted graph we can then apply the Min Cut Max Flow algorithm after connecting vertices to the virtual source and sink as previously described. 


\section{Tests}

Being the conveyer of intuition, we provide some examples. First an image is converted into a greyscale representation using the characteristic function previously described 

\begin{figure}[H]
	\centering
	\includegraphics[scale=0.25]{circlepic.png}
	\hspace{2em}
	\includegraphics[scale=0.35]{circlesgraph2.png}
	\caption{Representation of a image as a set of points in $\mathbb{R}^2$, blue points are then used as vertices in the graph.}
\end{figure}

Each pixel in the image is thresholded to produce a greyscale version, the pixels passing the threshold are then converted in to cartesian coordinates which represent the embedding of the graph in $\mathbb{R}^2$. 

In practice the complete graph is created using these nodes and then pruned down to the nearest neighbors, with 24 being used for the results here. Using the previously described algorithm we can then compute the flatnorm for these graphs using the points colored in blue as $\Omega$ and the full image as the background set $\Sigma$. The flat norm with scale provides a bounded curvature approximation to the set $\Omega$ according to the choice of $\lambda$. Our circle example highlights that for $\lambda$ large we will be able to reconstruct the set exactly but as $\lambda$ becomes smaller, circles of radius $r$ with $r < \frac{2}{\lambda}$ will begin to disappear in the reconstruction.

\begin{figure}[H]
	\centering
	\includegraphics[scale=0.5]{circleslamb.png}
	\caption{The flat norm reconstruction of the image, as lambda decreases higher curvature regions are removed.}
\end{figure}


Since weights are chosen according to the $F_CE(\Sigma,\Omega)$ functional one is able to recover the perimeter of $\Sigma$. This is achieved by summing over weights for edges that cross from $\Sigma$ into the background image $\Omega$. We do this in 2D and 3D for the circle and sphere respectively, and get the following approximations.

\begin{figure}[H]
	\centering
	\begin{tabular}{|c|c|c|c|}
		\hline
		&  Grid Size & Perimeter Estimate Relative Error  \\
		\hline
		$\mathbb{S}^1$ & 30x30 &  2.6\%  \\
		\hline
		$\mathbb{S}^2$ & TBD & TBD    \\
		\hline
	\end{tabular}
	\caption{Estimates from the perimeter term for the volume of spheres.}
\end{figure}

By creating the a graph that represents the epigraph, this perimeter term gives the arc length of the graph over its domain. We estimate the arclength of $x^2$ and the functions $f_a(x) = \sin(2\pi ax)$ for several choice of $a$. The accuracy of the perimeter estimation becomes poor as frequency becomes higher, indicating a relationship between curvature and the accuracy of the flat norm approximation.

\begin{figure}[H]
	\centering
	\begin{tabular}{|c|c|c|c|}
		\hline
		&  Grid Size & Arc Length Estimate Relative Error  \\
		\hline
		$x^2$ & 100x100 &  0.22\%  \\
		\hline
		$f_{0.5}(x)$ &  & 8\%    \\
		\hline
		$f_{1}(x)$ &  & 30\%    \\
		\hline
	\end{tabular}
	\caption{Estimates of arclength done with 8 nearest neighbors approximation.}
\end{figure}

\section{Edge Cases}

A limitation of the current approach is that for sets $\Omega$ sufficiently close to the boundary of the image pathological behavior occurs.



\bibliographystyle{plain} % We choose the "plain" reference style
\bibliography{refs} % Entries are in the refs.bib file

\end{document}
